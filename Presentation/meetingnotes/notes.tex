% Template last modified by Jake Hart; please contact course staff if you have any questions regarding using this template

\documentclass{cisXXX} % You must have the cisXXX .cls file in your project or working directory (i.e. the same directory as this document) 

\HWauthor{Wenda Zhao, Sasan Vakili}{} % Put your name and Penn email on this line
\HWno{1} % Enter the number of the homework you are working on
\HWcourse{AER1513H} % Enter the course department and number here
%\HWpartner{Paul Brown} % If your class allows group work, put your partners here
%\HWpartner{Amelia Earhart} % Otherwise, delete or comment these lines 
\usepackage{amsmath}
\usepackage{listings}
\usepackage{tabularx}
\usepackage{array}
\usepackage{bm}
\definecolor{mygreen}{rgb}{0,0.6,0}
%\usepackage{\xcolor}
\renewcommand\arraystretch{1.2}

\begin{document}
\maketitle
%\HWproblem
The motion model for quadrotor we define in our project is 
\begin{equation}
\begin{bmatrix}
x_k \\
y_k \\
z_k
\end{bmatrix}=\begin{bmatrix}
x_{k-1} \\
y_{k-1} \\
z_{k-1}
\end{bmatrix}+T\begin{bmatrix}
v_{x(k-1)} \\
v_{y(k-1)} \\
v_{z(k-1)} 
\end{bmatrix}+\frac{1}{2}T^2\left(R^T_{3\times3}\begin{bmatrix}
a_x   \\
a_y   \\
a_z
\end{bmatrix}-g\begin{bmatrix}
1  \\
1  \\
1
\end{bmatrix}\right)
\end{equation}

\begin{equation}
\begin{bmatrix}
v_{x(k)} \\
v_{y(k)} \\
v_{z(k)} 
\end{bmatrix}=\begin{bmatrix}
v_{x(k-1)} \\
v_{y(k-1)} \\
v_{z(k-1)} 
\end{bmatrix}+T\left(R^T_{3\times3}\begin{bmatrix}
a_x   \\
a_y   \\
a_z
\end{bmatrix}-g\begin{bmatrix}
1  \\
1  \\
1
\end{bmatrix}\right)
\end{equation}
where $(x,y,z)^T$ is the position of quadrotor and $(v_X,v_y,v_z)$ is the velocity. $T$ is the sampling time period, $g$ is the gravity and $R^T_{3\times3}$ is the rotation matrix from body frame to global frame.The motion model can be further expressed into state space equation.
\begin{equation}
\begin{bmatrix}
x_k \\
y_k \\
z_k \\
v_{x(k)} \\
v_{y(k)} \\
v_{z(k)} 
\end{bmatrix}=\begin{bmatrix}
1  & 0  & 0  & T  & 0  & 0 \\
0  & 1  & 0  & 0  & T  & 0 \\
0  & 0  & 1  & 0  & 0  & T \\
0  & 0  & 0  & 1  & 0  & 0 \\
0  & 0  & 0  & 0  & 1  & 0 \\
0  & 0  & 0  & 0  & 0  & 1 \\
\end{bmatrix}
\begin{bmatrix}
x_{k-1} \\
y_{k-1} \\
z_{k-1} \\
v_{x(k-1)} \\
v_{y(k-1)} \\
v_{z(k-1)} 
\end{bmatrix}+\begin{bmatrix}
\frac{1}{2}T^2  & 0     & 0   \\
0      &\frac{1}{2}T^2  & 0   \\
0      & 0  & \frac{1}{2}T^2  \\
T      & 0              & 0   \\
0      & T              & 0   \\
0      & 0              & T
\end{bmatrix}\left( R^T_{3\times3}\begin{bmatrix}
a_x  \\
a_y  \\
a_z
\end{bmatrix}-g\begin{bmatrix}
1  \\
1  \\
1
\end{bmatrix}
\right)
\end{equation}
Rotation matrix can be expressed by unit quaternion $q=(q_0, \bm{\vec{q_v}})^T$
$$
R^T_{3\times3}=(2q_0^2-1)\bm{1}_{3\times3}+2\bm{\vec{q_v}}\bm{\vec{q_v}}^T-2q_0\bm{\vec{q_v}}^{\times}
$$
where $\bm{\vec{q_v}}^{\times}$ is the skew-symmetric cross product matrix of $\bm{\vec{q_v}}$.

Observation model
\begin{equation}
\begin{bmatrix}
z_k     \\
v_{x(k)}  \\
v_{y(k)}
\end{bmatrix} = \begin{bmatrix}
0  & 0  & 1  & 0  & 0  & 0    \\
0  & 0  & 0  & 1  & 0  & 0    \\
0  & 0  & 0  & 0  & 1  & 0    
\end{bmatrix}\begin{bmatrix}
x_k \\
y_k \\
z_k \\
v_{x(k)} \\
v_{y(k)} \\
v_{z(k)}
\end{bmatrix}
\end{equation}

Data source:
\begin{itemize}
\item Input: unit quaternion $q=(q_0, \bm{\vec{q_v}})^T$ from onboard extended Kalman filter. (collected)

\item Input: acceleration $a=(a_x, a_y, a_z)^T$ from onboard IMU. (collected)
\item Measurements: $(z, v_x,v_y)^T$ from flowdeck.(collected)
\end{itemize}


\end{document}






